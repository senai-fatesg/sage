\chapter{Escopo do Projeto}
\label{cap:escopodoprojeto}

Este documento de modelagem de Sistema prev� uma vis�o completa dos modelos do
sistema SAGE, sendo produzido e utilizado pela equipe de desenvolvimento para
documentar os requisitos, modelos, tecnologias e arquitetura do sistema. 

\section{Situa��o Atual}
\label{sec:situacaoatual}

Atualmente seus processos tem sido feito mediante planilha impressa e de forma
manual, relat�rios de presen�a do estagi�rio e documentos de acompanhamento,
entre outros tipos de controle.  

\subsection{Motiva��o}
\label{subsec:motivacao}

\subsection{Problema}
\label{subsec:problema}

\subsection{Hip�tese}
\label{subsec:hipotese}

\subsection{Justificativa}
\label{subsec:justificativa}

\section{Objetivos}
\label{sec:objetivo}

\subsection{Objetivos Gerais}
\label{subsec:objetivosgerais}

O projeto tem como objetivo facilitar e inovar o gerenciamento das atividades
referentes ao controle de est�gio de institui��es, retirar o processo manual
feito pela coordena��o de est�gio e pelas empresas. Ganho de tempo, facilidade
de acesso a informa��o, elabora��o do programa de est�gio de forma mais rapida,
possibilitando a automa��o do processo de est�gio. 

\subsection{Objetivos Espec�ficos}
\label{subsec:objetivosespificos}

Desenvolver um sistema para a melhoria do processo de gest�o do est�gio, inovar
os processos produtivos, fazer controle de todo o processo de cadastro das
empresas, divulga��o das vagas de est�gio, sele��o dos alunos aptos a
participarem do processo seletivo, matr�cula dos alunos no programa de est�gio e 
tamb�m possibilitar todo o processo de avalia��o do desempenho do aluno no
est�gio, gerar relat�rios.
Al�m das funcionalidades descritas acima, o sistema ter� tamb�m a constru��o de
um banco de dados com todas as informa��es referente ao est�gio que
posteriormente poder� ser consultado.

\section{Cronograma}
\label{sec:cronograma}
