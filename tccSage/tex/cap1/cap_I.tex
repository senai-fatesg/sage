\chapter{Introdu��o}
\label{cap:introducao}
%COME�AR COM A DESCRICAO DO PROBLEMA E TERMINA-LA COM OBJETIVO
%CRIAR HIPOTESES DO TIPO: SER� QUE?
Desenvolvemos uma aplica��o Web para facilitar e inovar o gerenciamento das
atividades de acompanhamento e gerenciamento de est�gio supervisionado para
institui��es de ensino, possibilitando a automa��o da autogest�o do
programa de est�gio.
Pelo qual esse processo de gest�o de est�gio � feita manualmente pela
coordena��o do est�gio, no qual perde-se muito tempo no controle e avalia��o
devido ao processo manual.

Este processo inicia atrav�s da empresa divulgando a vaga de est�gio,
em seguida com os alunos devidamente selecionados para participarem do
processo e posteriormente a matr�cula no est�gio.

Como vimos todo esse processo e feito de forma manual.


\section{Finalidade}
\label{sec:finalidade}
%CONTEXTUALIZAR, TRAZER O PROBLEMA
%TRAGO A HIPOTESE E CONTINUO: POR ISSO QUE FIZ \ldots ISSO E ISSO
%EXPERIMENTOS EU RESPONDO AS HIPOTESES
Este documento apresenta a modelagem do sistema SAGE $-$ Sistema de 
Acompanhamento e Gerenciamento de Est�gio.

Apresentaremos neste trabalho, sobre as fases de desenvolvimento de software,
que v�o o levantamento de requisitos at� o produto completo. O p�blico alvo
deste documento inclui as pessoas envolvidas no desenvolvimento do sistema
(analistas e desenvolvedores) todos os envolvidos no projeto.


