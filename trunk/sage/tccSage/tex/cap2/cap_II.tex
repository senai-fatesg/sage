\chapter{Escopo do Projeto}
\label{cap:escopodoprojeto}

Este documento de modelagem de Sistema prevê uma visão completa dos modelos do
sistema SAGE, sendo produzido e utilizado pela equipe de desenvolvimento para
documentar os requisitos, modelos, tecnologias e arquitetura do sistema. 

\section{Situação Atual}
\label{sec:situacaoatual}

Atualmente seus processos tem sido feito mediante planilha impressa e de forma
manual, relatórios de presença do estagiário e documentos de acompanhamento,
entre outros tipos de controle.  

\subsection{Motivação}
\label{subsec:motivacao}

\subsection{Problema}
\label{subsec:problema}

\subsection{Hipótese}
\label{subsec:hipotese}

\subsection{Justificativa}
\label{subsec:justificativa}

\section{Objetivos}
\label{sec:objetivo}

\subsection{Objetivos Gerais}
\label{subsec:objetivosgerais}

O projeto tem como objetivo facilitar e inovar o gerenciamento das atividades
referentes ao controle de estágio de instituições, retirar o processo manual
feito pela coordenação de estágio e pelas empresas. Ganho de tempo, facilidade
de acesso a informação, elaboração do programa de estágio de forma mais rapida,
possibilitando a automação do processo de estágio. 

\subsection{Objetivos Específicos}
\label{subsec:objetivosespificos}

Desenvolver um sistema para a melhoria do processo de gestão do estágio, inovar
os processos produtivos, fazer controle de todo o processo de cadastro das
empresas, divulgação das vagas de estágio, seleção dos alunos aptos a
participarem do processo seletivo, matrícula dos alunos no programa de estágio e 
também possibilitar todo o processo de avaliação do desempenho do aluno no
estágio, gerar relatórios.
Além das funcionalidades descritas acima, o sistema terá também a construção de
um banco de dados com todas as informações referente ao estágio que
posteriormente poderá ser consultado.

\section{Cronograma}
\label{sec:cronograma}
