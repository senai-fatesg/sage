\chapter{Fundamentação Teórica}
\label{cap:fundamentacaoTeorica}

\section{Engenharia de Software}
\label{sec:engenhariaSoftware}

A engenharia de software é uma tecnologia em camadas (1 - Ferramentas; 2
- Métodos; 3 - Processo; 4 - Foco na qualidade).
Qualquer abordagem de engenharia (inclusive engenharia de software) deve estar fundamentada em um
comprometimento organizacional com a qualidade. A gestão de qualidade total Seis
Sigma e filosofias similares promovem uma cultura de aperfeiçoamento contínuo de
processoes, e é esta cultura que, no final das contas, leva ao desenvolvimento
de abordagens cada vez mais efetivas na engenharia de software. A pedra
fundamental que sustenta a engenharia de software é o foco na qualidade \cite{Pressman:2011}.

\section{Gestão de Projetos}
\label{sec:gestaoProjeto}

Abrange uma série de ferramentas e técnicas utilizadas por pessoas para 
descrever, organizar e monitorar o andamento das atividades do projeto. Os
gerentes de projeto são os responsáveis pela administração dos processos 
envolvidos e pela aplicação das ferramentas e técnicas necessárias ao 
cumprimento das atividades do projeto. Todo projeto é composto por processos,
por mais irrelevante que seja a abordagem empregada. 

A Gestão de Projetos consiste na aplicação de conhecimento, competências, ferramentas e técnicas às
atividades do projeto, com vista ao cumprimento dos requisitos em pauta. É
responsabilidade do gerente de projeto assegurar que tais técnicas sejam
utilizadas e seguidas \cite{Heldman:2011}.

\section{Análise e Definição de Requisitos}
\label{sec:analiseDefinicaoRequisitos}

A análise e definição dos requisitos de software são atividades no
desenvolvimento do sistema onde serão levantados e definidos os elementos que
irão compor o software e as restrições associadas a eles. A análise deve estabelecer
o relacionamento entre estes objetivos e restrições e a especificação precisa do
software. Nestas atividades, o contato com o cliente é efetivo e constante, uma vez
que é necessário extrair as regras do negócio.

É o processo de observação e levantamento dos elementos do domínio no
qual o sistema será introduzido. Deve-se identificar as pessoas, as atividades,
informações do domínio para que se possa decidir o que deverá ser informatizado.

\subsection{Principais Técnicas de Análise e Levantamento de Requisitos}
\label{subsec:principaisTecnicasAnalise}

Existem algumas técnicas de levantamento de requisitos que são utilizadas,
reuniões, questionários, entrevistas individuais, sessões, etc. Os analistas podem
empregar uma ou várias técnicas para elicitar os requisitos dos clientes, isto envolve
situações tais como organizar entrevistas ou grupos focais (workshops) e a criação
de lista de requisitos. Técnicas mais modernas incluem prototipação, e casos de
uso, onde o analista irá aplicar uma combinação de métodos para estabelecer os
requisitos exatos de seus stakeholders, tal que um sistema que atenda as
necessidades do negócio seja produzido.

\subsection{Tipos de Requisitos}
\label{subsec:tiposRequisitos}

Os requisitos de um sistema são descrições do sistema, ou seja, uma declaração
abstrata de alto nível de um serviço que o sistema deve fornecer ou uma restrição 
do sistema \cite{Sommerville:2008}.

\subsubsection{Requisitos Funcionais}
\label{subsec:RequisitosFuncionais}

Os requisitos funcionais de um sistema estão totalmente ligados ao software, o
proposito desses requisitos é descrever uma função do sistema, ou seja, 
descrever detalhadamente seu conjunto de entradas, seu comportamento e as  suas
saídas \cite{Sommerville:2008}

\subsubsection{Requisitos Não-Funcionais}
\label{subsec:RequisitosNaoFuncionais}

Os requisitos não funcionais são aqueles não diretamente relacionados às funções
específicas fornecidas pelo sistema. Esses requisitos surgem devido às 
necessidades do usuário, tais como: segurança, confiabilidade, usabilidade, 
desempenho, entre outras necessidades que forem necessárias e que não estão 
ligadas a funções do sistema \cite{Sommerville:2008}.

\subsection{Protótipos}
\label{subsec:prototipacao}

A prototipação é uma parte importantíssima no processo de desenvolvimento de
software, pois é a fase em que o desenvolvedor começa a modelar os elementos do
sistema, assim ficando até mais fácil na hora de começar a realmente 
desenvolver o sistema. Também pode ser útil para o usuário final, já que se 
trata de elementos do sistema com quais os usuários terão contato
\cite{Pressman:2010}. Tornando mais compreensível, para o usuário final, a
transformação dos requisitos coletados no sistema propriamente dito. É uma forma de o usuário 
validar os requisitos coletados.

Além de validar os requisitos já coletados com o cliente, o protótipo pode 
auxiliar na identificação de novos requisitos que no momento da coleta não foram 
observados. 

É também uma forma de reduzir os custos, já que os erros e omissões 
durante a fase de levantamento dos requisitos, podem ser muito onerosos quando 
necessitam de correções em outras fases. 

\subsection{Validação de Requisitos}
\label{subsec:validacaoRequisitos}

A validação de requisitos é uma parte muito importante no processo de
desenvolvimento de software, pois é nessa parte que o engenheiro de software,
clientes, usuários e outros interessados examinam a especificação procurando 
por erros de conteúdo ou de interpretação, inconsistências, informações 
omissas, entre outras coisas \cite{Pressman:2010}.

\subsection{Linguagem Unificada de Modelagem - UML}
\label{subsec:uml}

A UML(Unified Modeling Language - linguagem de modelagem unificada) é "uma
linguagem-padrão para descrever/documentar projeto de software. A UML pode ser
usada para visualizar, especificar, construir e documentar os artefatos de um
sistema de software-intensivo". Em outras palavras, assim como os arquitetos
criam plantas e projetos para ser usados por uma empresa de construção, os
arquitetos de software criam diagramas UML para ajudar os desenvolvedores de
software a construir o software \cite{Pressman:2011}.

Na UML o significado de “classe” pode ser observada em diferentes
perspectivas, como segue:

• Classe Conceitual – coisa ou conceito do mundo real

• Classe de Software – classe que representa uma perspectiva de
especificação ou implementação de um elemento de software, independente
do processo ou método.

• Classe de Implementação – classe implementada em uma linguagem
orientada a objetos especifica, como a Java.

\subsection{Problemas}
\label{subsec:problemas}

A atividade de levantamento de requisitos nem sempre traz um cenário
perfeito ou amigável para o analista de sistemas. Muitas das vezes as pessoas que
serão envolvidas no processo não estão dispostas a colaborar, isso por diversos
motivos, tais como o medo de perder o emprego para uma máquina ou medo de que
o software leve a tona uma realidade que não é a que eles (o usuário especialista)
queriam estar mostrando da empresa. Seja qual for o motivo da não colaboração de
nossos stakeholders, ainda assim nos deparamos com alguns problemas críticos,
tais como: nem mesmo os “especialistas” daquele departamento ou setor da
empresa em que vai ser construído o software sabe ao certo como é o fluxo correto
de trabalho, como é a rotina do seu departamento.

\subsubsection{Problemas com Stakeholders}
\label{subsec:problemasStakeholders}

No processo de levantamento de requisitos com os stakeholders podem
aparecer alguns problemas que impeçam ou diminuam a clareza e obtenção dos
requisitos, os principais são:

• Usuários não sabem o que eles querem.

• Usuários que não querem concluir a escrita do conjunto de requisitos.

• Comunicação com o usuário é lenta.

• Os usuários freqüentemente não participam nas revisões ou são
incapazes de fazer isto.

• Os usuários são tecnicamente poucos sofisticados.

• Os usuários não entendem do desenvolvimento de processo.

Isto deve levar a situações onde os requisitos do usuário continuam mudando
mesmo quando o desenvolvimento do sistema ou produto já se iniciou.

\subsubsection{Problemas com Engenheiros/Desenvolvedores}
\label{subsec:problemasEngenheiros}

No processo de análise de requisitos podem aparecer alguns problemas
como:

• Pessoal técnico e usuários finais têm vocabulários diferentes.
Conseqüentemente, eles podem acreditar que estão em perfeito acordo
até que o produto final seja entregue.

• Engenheiros e desenvolvedores tentam ajustar os requisitos para um
sistema existente ou modelo, em vez de desenvolver um sistema
específico que atenda as necessidades do cliente.

• A análise é freqüentemente conduzida por engenheiros ou programadores,
ao invés de pessoal com habilidade e domínio do conhecimento para
compreender as necessidades dos clientes.

\subsection{Rastreabilidade de Requisitos}
\label{subsec:rastreabilidade}

O rastreamento de requisitos é utilizado para prover uma gestão desses requisitos, 
e para cada requisito é atribuído um identificador. Uma vez identificados os requisitos, 
tabelas de rastreamento são desenvolvidas. Cada uma dessas tabelas relaciona um ou mais 
requisitos do sistema ou de seu ambiente. 
Em muitos casos essas tabelas são mantidas como parte do banco de dados de requisitos,
de modo que elas possam ser encontradas com mais facilidade e serem feitas alterações 
caso seja necessário \cite{Pressman:2010}.

\subsection{O Uso de Ferramentas para a Análise e Gerênciamento de Requisitos}
\label{subsec:ferramentasAnalise}

As empresas possuem a necessidade de se medir a sua capacidade de
gerenciar o processo de Análise de requisitos. Para isto, existem no mercado
ferramentas que utilizam modelo CMMI (Capability Maturity Model Integration), que é
um modelo gerencial que organiza as melhores práticas existentes, embora os
padrões e as práticas que são aplicáveis não sejam completamente definidos.

Dentre essas ferramentas de gerenciamento de requisitos de software
destacam-se:

• Borland CaliberRM

É uma ferramenta de software corporativa para o gerenciamento de requisitos
que facilita a colaboração, a análise de impacto e a comunicação, permitindo que as
equipes de software avancem o projeto com maior precisão e previsibilidade.

A arquitetura aberta do Borland CaliberRM permite que o analista de sistemas
conecte os requisitos do software a uma variedade de artefatos utilizados no ciclo de
vida, possibilitando assim uma rastreabilidade de requisitos de ponta a ponta.

\subsection{Qualidade na fase de Análise de Requisitos}
\label{subsec:qualidadeAnalise}

Dentro da Engenharia de Software existe uma busca constante no que tange
a qualidade no desenvolvimento de software, algumas empresas utilizam o modelo
RUP, que é um modelo gerencial que organiza a
melhores práticas existentes e que mede a qualidade por meio da maturidade da
capacidade dos processos de software.

O que se nota é que se deve focalizar e investir no processo de melhoria
contínua visando à qualidade dos requisitos ou da forma de gerência dos requisitos
que irão compor o software.

O RUP é um framework genérico e complexo,
pois visa atender todos os tipos de projetos de desenvolvimento de software. Toda
disciplina do RUP deve ser analisada e customizada de acordo com as
necessidades específicas do projeto antes de sua implantação. Um framework de
maneira simples, é um conjunto de objetos extensível para funções relacionadas,
que visa fornecer uma implementação para as funções básicas e invariantes e inclui
um mecanismo para permitir que o desenvolvedor se conecte às diversas funções
ou as estenda.

\section{Arquitetura de Software}
\label{sec:arquiteturaSw}

A Arquitetura de Software não esta relacionada a uma fase do desenvolvimento do
sistema em si. Essa é a fase em que é feita a representação que permite o engenheiro 
de software analisar a efetividade do projeto em satisfazer a seus requisitos 
declarados, considerar alternativas para realizar mudanças arquiteturais e
reduzir riscos no processo de construção do software \cite{Pressman:2010}.
